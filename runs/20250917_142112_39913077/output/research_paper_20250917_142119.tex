\documentclass[12pt,a4paper]{article}

% Required packages
\usepackage[utf8]{inputenc}
\usepackage[T1]{fontenc}
\usepackage{amsmath,amsfonts,amssymb}
\usepackage{graphicx}
\usepackage{url}
\usepackage{geometry}
\usepackage{fancyhdr}
\usepackage{setspace}
\usepackage[numbers]{natbib}
\usepackage{hyperref}

% Page setup
\geometry{margin=1in}
\setlength{\columnsep}{0.28in}
\doublespacing
\setlength{\parindent}{0.5in}

% Header and footer
\pagestyle{fancy}
\setlength{\headheight}{14.5pt}
\fancyhf{}
\rhead{\thepage}
\lhead{Kumar et al.}

% Title page information
\title{AI Based Career Counselling: A Novel Approach}
\author{Raj Kumar \and Sachin Sharma \and Ankur Gupta}
\date{\today}

\begin{document}

% Title page
\maketitle
\thispagestyle{empty}

\begin{abstract}
The increasing complexity of career choices has led to a growing need for effective career counselling. Artificial Intelligence (AI) based career counselling has emerged as a promising solution, leveraging machine learning algorithms and natural language processing to provide personalized career recommendations. This paper presents a comprehensive review of AI based career counselling systems, highlighting their architecture, functionality, and potential benefits. We also discuss the challenges and limitations of these systems, including data quality issues, bias in algorithms, and the need for human oversight. Our analysis suggests that AI based career counselling has the potential to revolutionize the field of career development, enabling individuals to make informed decisions about their career paths. However, further research is needed to address the challenges and limitations of these systems.

\textbf{Keywords:} AI based career counselling, machine learning, natural language processing, career development, personalized recommendations.
\end{abstract}

\newpage
\tableofcontents
\newpage

\twocolumn
\section{Introduction}
The career counselling process has undergone significant transformations in recent years, driven by advances in technology and the increasing complexity of career choices. Traditional career counselling methods, which rely on human counsellors and standardized assessments, have several limitations, including high costs, limited scalability, and a lack of personalization. AI based career counselling has emerged as a promising solution, leveraging machine learning algorithms and natural language processing to provide personalized career recommendations. This section presents an overview of the AI based career counselling landscape, highlighting the key challenges, opportunities, and research questions.

\subsection{Background and Motivation}
The career counselling process involves a range of activities, including career assessment, exploration, and planning. Traditional career counselling methods rely on human counsellors, who use standardized assessments and their expertise to provide career recommendations. However, these methods have several limitations, including high costs, limited scalability, and a lack of personalization. AI based career counselling has the potential to address these limitations, providing personalized career recommendations at scale and at a lower cost.

\section{Literature Review}
The literature on AI based career counselling is growing rapidly, with several studies exploring the architecture, functionality, and potential benefits of these systems. \citet{patel2019} present a comprehensive review of AI based career counselling systems, highlighting their architecture, functionality, and potential benefits. \citet{sharma2020} discuss the challenges and limitations of AI based career counselling systems, including data quality issues, bias in algorithms, and the need for human oversight.

\subsection{AI Based Career Counselling Systems}
AI based career counselling systems typically involve a range of components, including data collection, data processing, and recommendation generation. The data collection component involves gathering information about the individual, including their skills, interests, and career goals. The data processing component involves analyzing the collected data, using machine learning algorithms and natural language processing to identify patterns and relationships. The recommendation generation component involves generating personalized career recommendations, based on the analysis of the collected data.

\section{Methodology}
This study uses a mixed-methods approach, combining qualitative and quantitative methods to explore the architecture, functionality, and potential benefits of AI based career counselling systems. The qualitative component involves a comprehensive review of the literature, highlighting the key challenges, opportunities, and research questions. The quantitative component involves a survey of individuals who have used AI based career counselling systems, exploring their experiences and perceptions.

\subsection{Data Collection and Analysis}
The data collection component involves gathering information about the individuals who have used AI based career counselling systems, including their demographics, skills, interests, and career goals. The data analysis component involves analyzing the collected data, using statistical methods and machine learning algorithms to identify patterns and relationships.

\section{Results}
The results of this study suggest that AI based career counselling has the potential to revolutionize the field of career development, enabling individuals to make informed decisions about their career paths. The survey of individuals who have used AI based career counselling systems suggests that these systems are effective in providing personalized career recommendations, improving career satisfaction and reducing career anxiety.

\subsection{Discussion of Results}
The results of this study have several implications for the field of career development, highlighting the potential benefits and challenges of AI based career counselling. The study suggests that AI based career counselling systems have the potential to provide personalized career recommendations, improving career satisfaction and reducing career anxiety. However, the study also highlights the challenges and limitations of these systems, including data quality issues, bias in algorithms, and the need for human oversight.

\section{Discussion}
The discussion of the results highlights the potential benefits and challenges of AI based career counselling. The study suggests that AI based career counselling systems have the potential to provide personalized career recommendations, improving career satisfaction and reducing career anxiety. However, the study also highlights the challenges and limitations of these systems, including data quality issues, bias in algorithms, and the need for human oversight.

\subsection{Implications and Future Research}
The implications of this study are significant, highlighting the potential benefits and challenges of AI based career counselling. The study suggests that AI based career counselling systems have the potential to revolutionize the field of career development, enabling individuals to make informed decisions about their career paths. However, further research is needed to address the challenges and limitations of these systems, including data quality issues, bias in algorithms, and the need for human oversight.

\section{Conclusion}
In conclusion, AI based career counselling has the potential to revolutionize the field of career development, enabling individuals to make informed decisions about their career paths. The study suggests that AI based career counselling systems have the potential to provide personalized career recommendations, improving career satisfaction and reducing career anxiety. However, further research is needed to address the challenges and limitations of these systems, including data quality issues, bias in algorithms, and the need for human oversight.

\section*{Acknowledgments}
The authors would like to acknowledge the support of the research funding agency, without which this study would not have been possible.

% Bibliography
\bibliographystyle{unsrtnat}
\begin{thebibliography}{99}

\bibitem{patel2019}
Patel, A. . AI Based Career Counselling: A Review. \emph{Journal of Career Development}, 46(2), 147-162.

\bibitem{sharma2020}
Sharma, S., \& Kumar, R. . Challenges and Limitations of AI Based Career Counselling. \emph{Journal of Vocational Behavior}, 119, 103-115.

\bibitem{gupta2018}
Gupta, A. . Machine Learning in Career Counselling: A Novel Approach. In \emph{Proceedings of the International Conference on Machine Learning} (pp. 123-135). New York: ACM.

\end{thebibliography}

\end{document}