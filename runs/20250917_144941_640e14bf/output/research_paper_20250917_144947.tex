\documentclass[12pt,a4paper]{article}

% Required packages
\usepackage[utf8]{inputenc}
\usepackage[T1]{fontenc}
\usepackage{amsmath,amsfonts,amssymb}
\usepackage{url}
\usepackage{geometry}
\usepackage{fancyhdr}
\usepackage{setspace}
\usepackage[numbers]{natbib}
\usepackage{hyperref}

% Page setup
\geometry{margin=1in}
\setlength{\columnsep}{0.28in}
\doublespacing
\setlength{\parindent}{0.5in}

% Header and footer
\pagestyle{fancy}
\setlength{\headheight}{14.5pt}
\fancyhf{}
\rhead{\thepage}
\lhead{}

% Title page information
\title{The Impact of Social Media on Interpersonal Relationships}
\author{Emily J. Smith \and John D. Lee \and Sarah K. Taylor}
\date{\today}

\begin{document}

% Title page
\maketitle
\thispagestyle{empty}

\begin{abstract}
\setlength{\parindent}{0pt}
\noindent
The proliferation of social media has transformed the way we interact and maintain relationships. This study investigates the complex dynamics between social media usage and the quality of interpersonal relationships. We surveyed 500 participants and analyzed the data using a regression model to identify significant predictors of relationship satisfaction. Our findings indicate that excessive social media use is associated with decreased relationship satisfaction, particularly when individuals engage in social comparison and experience feelings of jealousy. The results are discussed in the context of social penetration theory and the social exchange theory.

\textbf{Keywords:} social media, interpersonal relationships, relationship satisfaction, social comparison, jealousy
\end{abstract}

\newpage
\tableofcontents
\newpage

\twocolumn
\section{Introduction}
The advent of social media has revolutionized the way we communicate and maintain relationships. Platforms such as Facebook, Twitter, and Instagram have become integral to our daily lives, enabling us to connect with others across geographical distances. However, concerns have been raised about the potential impact of social media on the quality of our relationships. This study aims to investigate the complex dynamics between social media usage and interpersonal relationships, with a focus on the predictors of relationship satisfaction.

\subsection{Research Questions}
Our study addresses the following research questions: (1) Is there a significant association between social media usage and relationship satisfaction? (2) Do social comparison and jealousy mediate the relationship between social media use and relationship satisfaction?

\section{Literature Review}
Existing research on the impact of social media on relationships has yielded mixed findings. Some studies have reported a positive association between social media use and relationship satisfaction \citep{burke2010social}, while others have found a negative correlation \citep{király2019mental}. The social penetration theory suggests that relationships progress through stages of intimacy, with self-disclosure being a critical factor \citep{altman1973social}. Social media can facilitate self-disclosure, but it can also lead to social comparison and feelings of jealousy, which can negatively impact relationship satisfaction.

\subsection{Theoretical Framework}
Our study is grounded in the social exchange theory, which posits that relationships are maintained when the benefits outweigh the costs \citep{emerson1976social}. We argue that social media use can affect the perceived benefits and costs of a relationship, influencing relationship satisfaction.

\section{Methodology}
We conducted an online survey of 500 participants, recruited through social media platforms and online forums. The survey included measures of social media use, relationship satisfaction, social comparison, and jealousy. We used a regression model to analyze the data, with relationship satisfaction as the outcome variable.

\subsection{Measures}
Social media use was measured using a scale assessing frequency and duration of use. Relationship satisfaction was assessed using the Relationship Assessment Scale \citep{hendrick1988relationship}. Social comparison and jealousy were measured using scales adapted from existing research \citep{burke2010social}.

\section{Results}
Our regression analysis revealed a significant negative association between social media use and relationship satisfaction ($\beta = -0.23$, $p < 0.01$). Social comparison and jealousy mediated this relationship, with a significant indirect effect ($\beta = -0.12$, $p < 0.05$). The model explained 30\% of the variance in relationship satisfaction.

\subsection{Model Specification}
The regression model was specified as follows:
\[ Y = \beta_0 + \beta_1X + \beta_2M + \epsilon \]
where $Y$ is relationship satisfaction, $X$ is social media use, $M$ is the mediator (social comparison and jealousy), and $\epsilon$ is the error term.

\section{Discussion}
Our findings suggest that excessive social media use is associated with decreased relationship satisfaction, particularly when individuals engage in social comparison and experience feelings of jealousy. These results are consistent with the social exchange theory, which posits that relationships are maintained when the benefits outweigh the costs.

\subsection{Implications}
The study has implications for individuals seeking to maintain healthy relationships in the digital age. By being aware of the potential risks of social media use, individuals can take steps to mitigate its negative effects.

\section{Conclusion}
This study contributes to our understanding of the complex dynamics between social media usage and interpersonal relationships. Our findings highlight the need for a nuanced approach to understanding the impact of social media on relationships.

\section*{Acknowledgments}
This research was supported by a grant from the Social Sciences and Humanities Research Council.

% Bibliography
\bibliographystyle{unsrtnat}
\begin{thebibliography}{99}

\bibitem{burke2010social}
Burke, M. . Social network activity and social well-being. \emph{Computers in Human Behavior}, 26(6), 1427-1435.

\bibitem{király2019mental}
Király, O. . Mental health and addictive behaviors in young adults: A systematic review of clinical and neurobiological findings. \emph{Journal of Behavioral Addictions}, 8(3), 537-553.

\bibitem{altman1973social}
Altman, I., \& Taylor, D. A. . \emph{Social penetration: The development of interpersonal relationships}. Holt, Rinehart and Winston.

\bibitem{emerson1976social}
Emerson, R. M. . Social exchange theory. \emph{Annual Review of Sociology}, 2, 335-362.

\bibitem{hendrick1988relationship}
Hendrick, S. S. . A generic measure of relationship satisfaction. \emph{Journal of Marriage and the Family}, 50(1), 93-98.

\end{thebibliography}

\end{document}