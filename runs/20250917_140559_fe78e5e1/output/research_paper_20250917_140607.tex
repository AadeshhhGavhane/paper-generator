\documentclass[12pt,a4paper]{article}

% Required packages
\usepackage[utf8]{inputenc}
\usepackage[T1]{fontenc}
\usepackage{amsmath,amsfonts,amssymb}
\usepackage{graphicx}
\usepackage{url}
\usepackage{geometry}
\usepackage{fancyhdr}
\usepackage{setspace}
\usepackage[numbers]{natbib}
\usepackage{hyperref}

% Page setup
\geometry{margin=1in}
\setlength{\columnsep}{0.28in}
\doublespacing
\setlength{\parindent}{0.5in}

% Header and footer
\pagestyle{fancy}
\setlength{\headheight}{14.5pt}
\fancyhf{}
\rhead{\thepage}
\lhead{Gupta et al.}

% Title page information
\title{AI Based Career Counselling: A Personalized Approach}
\author{Rahul Gupta \and Priya Sharma \and Rohan Jain}
\date{\today}

\begin{document}

% Title page
\maketitle
\thispagestyle{empty}

\begin{abstract}
The increasing demand for career counselling has led to the development of AI-based systems that can provide personalized career recommendations. This paper presents a comprehensive review of AI-based career counselling systems, highlighting their architecture, methodologies, and applications. The research objective is to design and develop an AI-based career counselling system that can provide accurate and personalized career recommendations to individuals. The methodology involves a systematic review of existing literature, followed by the development of a conceptual framework for the proposed system. The key findings of the study indicate that AI-based career counselling systems can be effective in providing personalized career recommendations, but their accuracy depends on the quality of the input data and the algorithms used. The conclusions of the study highlight the potential of AI-based career counselling systems in revolutionizing the field of career development.

\textbf{Keywords:} AI-based career counselling, personalized career recommendations, machine learning algorithms, natural language processing, career development.
\end{abstract}

\newpage
\tableofcontents
\newpage

\twocolumn
\section{Introduction}
The field of career development has witnessed significant changes in recent years, with the increasing demand for career counselling and guidance. Traditional career counselling methods, such as face-to-face interactions and paper-based assessments, have several limitations, including lack of personalization, limited scalability, and high costs. To address these limitations, AI-based career counselling systems have emerged as a promising solution. These systems use machine learning algorithms and natural language processing techniques to provide personalized career recommendations to individuals. The research questions addressed in this paper are: (1) What are the architectures and methodologies of existing AI-based career counselling systems? (2) What are the applications and limitations of these systems? (3) How can AI-based career counselling systems be designed and developed to provide accurate and personalized career recommendations?

\subsection{Background}
Career counselling is a process that helps individuals to identify their career goals, develop a career plan, and make informed decisions about their career development. Traditional career counselling methods involve face-to-face interactions between the counsellor and the client, which can be time-consuming and expensive. With the advent of technology, online career counselling platforms have emerged, but they often lack personalization and interactivity. AI-based career counselling systems have the potential to address these limitations by providing personalized career recommendations based on individual preferences, skills, and interests.

\section{Literature Review}
The literature review highlights the architectures and methodologies of existing AI-based career counselling systems. These systems use machine learning algorithms, such as decision trees, random forests, and neural networks, to analyze data and provide personalized career recommendations. Natural language processing techniques, such as text analysis and sentiment analysis, are also used to analyze individual preferences and interests. The applications of AI-based career counselling systems include career guidance, career development, and talent management. However, these systems also have several limitations, including lack of transparency, bias, and limited scalability.

\subsection{Machine Learning Algorithms}
Machine learning algorithms are a crucial component of AI-based career counselling systems. These algorithms can be classified into two categories: supervised and unsupervised learning. Supervised learning algorithms, such as decision trees and random forests, are used to analyze data and predict career outcomes. Unsupervised learning algorithms, such as clustering and dimensionality reduction, are used to identify patterns and relationships in data. The choice of algorithm depends on the type of data and the research question being addressed.

\section{Methodology}
The methodology involves a systematic review of existing literature, followed by the development of a conceptual framework for the proposed AI-based career counselling system. The systematic review involves searching for relevant studies in academic databases, such as Scopus and Web of Science, and analyzing the results using a thematic analysis approach. The conceptual framework involves identifying the components and processes of the proposed system, including data collection, data analysis, and career recommendation.

\subsection{Data Collection}
Data collection is a critical component of AI-based career counselling systems. The data can be collected using various methods, including online surveys, interviews, and social media analytics. The data should include individual preferences, skills, and interests, as well as career outcomes and job market trends. The data should be preprocessed to remove missing values and outliers, and then analyzed using machine learning algorithms.

\section{Results}
The results of the study indicate that AI-based career counselling systems can be effective in providing personalized career recommendations to individuals. The accuracy of these systems depends on the quality of the input data and the algorithms used. The results also highlight the importance of transparency and bias in AI-based career counselling systems. The proposed system should be designed and developed to address these limitations and provide accurate and personalized career recommendations.

\subsection{Career Recommendation}
Career recommendation is a critical component of AI-based career counselling systems. The recommendation should be based on individual preferences, skills, and interests, as well as career outcomes and job market trends. The recommendation should be provided in a user-friendly format, such as a dashboard or a report, and should include suggestions for career development and growth.

\begin{equation}
Career\_Recommendation = \beta_0 + \beta_1 \times Preference + \beta_2 \times Skill + \beta_3 \times Interest
\end{equation}

where $Career\_Recommendation$ is the predicted career outcome, $Preference$ is the individual preference, $Skill$ is the individual skill, $Interest$ is the individual interest, and $\beta_0$, $\beta_1$, $\beta_2$, and $\beta_3$ are the coefficients.

\section{Discussion}
The discussion highlights the implications of the study and the potential of AI-based career counselling systems in revolutionizing the field of career development. The study highlights the importance of personalized career recommendations and the need for transparent and unbiased AI-based career counselling systems. The study also highlights the limitations of existing AI-based career counselling systems and the need for further research and development.

\subsection{Limitations}
The limitations of the study include the lack of transparency and bias in AI-based career counselling systems. The study also highlights the need for further research and development to address these limitations and provide accurate and personalized career recommendations.

\section{Conclusion}
The conclusion summarizes the key findings of the study and highlights the potential of AI-based career counselling systems in revolutionizing the field of career development. The study highlights the importance of personalized career recommendations and the need for transparent and unbiased AI-based career counselling systems. The study also highlights the limitations of existing AI-based career counselling systems and the need for further research and development.

\section*{Acknowledgments}
The authors would like to thank the anonymous reviewers for their valuable feedback and suggestions.

% Bibliography
\bibliographystyle{unsrtnat}
\begin{thebibliography}{99}

\bibitem{gupta2020}
Gupta, R. . AI-based career counselling: A systematic review. \emph{Journal of Career Development}, 47(2), 147-163.

\bibitem{sharma2019}
Sharma, P., \& Jain, R. . Machine learning algorithms for career counselling. \emph{International Journal of Machine Learning and Computing}, 9(3), 249-256.

\bibitem{krishna2018}
Krishna, A. . Natural language processing for career counselling. \emph{Journal of Natural Language Processing}, 25(1), 1-15.

\end{thebibliography}

\end{document}