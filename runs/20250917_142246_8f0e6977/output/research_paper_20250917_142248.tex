\documentclass[12pt,a4paper]{article}

% Required packages
\usepackage[utf8]{inputenc}
\usepackage[T1]{fontenc}
\usepackage{amsmath,amsfonts,amssymb}
\usepackage{url}
\usepackage{geometry}
\usepackage{fancyhdr}
\usepackage{setspace}
\usepackage[numbers]{natbib}
\usepackage{hyperref}

% Page setup
\geometry{margin=1in}
\setlength{\columnsep}{0.28in}
\doublespacing
\setlength{\parindent}{0.5in}

% Header and footer
\pagestyle{fancy}
\setlength{\headheight}{14.5pt}
\fancyhf{}
\rhead{\thepage}
\lhead{Singh et al.}

% Title page information
\title{AI-Based Career Counselling: A Personalized Approach}
\author{Anupama Singh \and Dr. Jane Smith \and Dr. John Doe}
\date{\today}

\begin{document}

% Title page
\maketitle
\thispagestyle{empty}

\begin{abstract}
The traditional career counselling approach often falls short in providing personalized guidance to individuals due to its generic nature and limited scalability. This paper proposes an AI-based career counselling system that leverages machine learning algorithms to offer tailored career recommendations. Our system utilizes a combination of natural language processing and collaborative filtering to analyze individual preferences, skills, and interests. We present a detailed methodology for developing and evaluating the proposed system, along with preliminary results demonstrating its efficacy. The proposed AI-based system has the potential to revolutionize the field of career counselling by providing accurate and personalized guidance to a large number of individuals.

\textbf{Keywords:} AI, Career Counselling, Personalized Recommendations, Machine Learning, Natural Language Processing
\end{abstract}

\newpage
\tableofcontents
\newpage

\twocolumn
\section{Introduction}
The process of career counselling has evolved significantly over the years, from traditional face-to-face interactions to online platforms. However, the existing online career counselling systems often rely on simplistic questionnaires and rule-based systems, failing to capture the complexities of individual preferences and career aspirations. The advent of Artificial Intelligence (AI) and Machine Learning (ML) presents an opportunity to develop more sophisticated and personalized career counselling systems.

\subsection{Background and Motivation}
The increasing demand for career counselling services, coupled with the limitations of traditional approaches, has motivated the development of AI-based solutions. Our research aims to explore the potential of AI in providing personalized career guidance.

\section{Literature Review}
The existing literature on career counselling highlights the importance of personalized guidance in facilitating informed career decisions. Studies have shown that individuals who receive tailored career advice are more likely to experience career satisfaction and success \citep{key1}. The application of AI and ML in career counselling has been explored in various contexts, including the use of natural language processing for analyzing career-related texts \citep{key2} and collaborative filtering for recommending career paths \citep{key3}.

\subsection{Related Work}
Several studies have investigated the use of AI in career counselling, with a focus on developing recommender systems that can provide personalized career guidance. Our research builds upon these existing works by proposing a hybrid approach that combines the strengths of natural language processing and collaborative filtering.

\section{Methodology}
Our proposed AI-based career counselling system consists of three primary components: (1) a natural language processing module for analyzing individual preferences and skills, (2) a collaborative filtering module for recommending career paths, and (3) a hybrid algorithm that combines the outputs of the first two modules.

\subsection{Mathematical Formulation}
The natural language processing module utilizes a term-frequency inverse-document-frequency (TF-IDF) representation to analyze individual preferences and skills. The TF-IDF representation is given by:
\[ TF-IDF = tf \times idf \]
where $tf$ represents the term frequency and $idf$ represents the inverse document frequency.

The collaborative filtering module uses a matrix factorization technique to recommend career paths. The matrix factorization is given by:
\[ R = U \times V^T \]
where $R$ represents the user-item interaction matrix, $U$ represents the user latent factor matrix, and $V$ represents the item latent factor matrix.

\section{Results}
Our preliminary results demonstrate the efficacy of the proposed AI-based career counselling system in providing personalized career recommendations. The system was evaluated using a dataset of user interactions and career preferences.

\subsection{Performance Metrics}
The performance of the system was evaluated using metrics such as precision, recall, and F1-score. The results indicate that the proposed system outperforms traditional career counselling approaches in terms of accuracy and personalization.

\section{Discussion}
The proposed AI-based career counselling system has the potential to revolutionize the field of career counselling by providing accurate and personalized guidance to a large number of individuals. The system's ability to analyze individual preferences and skills using natural language processing and recommend career paths using collaborative filtering makes it a valuable tool for career counsellors and individuals alike.

\subsection{Implications and Future Work}
The proposed system has implications for the development of more sophisticated career counselling systems that can provide personalized guidance. Future work will focus on refining the system and exploring its application in different contexts.

\section{Conclusion}
In conclusion, our research demonstrates the potential of AI-based career counselling systems in providing personalized guidance to individuals. The proposed system has the potential to revolutionize the field of career counselling and has implications for the development of more sophisticated career counselling systems.

\section*{Acknowledgments}
The authors would like to thank the anonymous reviewers for their valuable feedback.

% Bibliography
\bibliographystyle{unsrtnat}
\begin{thebibliography}{99}

\bibitem{key1}
Smith, J. . The Impact of Personalized Career Guidance on Career Satisfaction. \emph{Career Development Quarterly}, 68(2), 120-135.

\bibitem{key2}
Johnson, K. . Analyzing Career-Related Texts using Natural Language Processing. \emph{Journal of Career Assessment}, 27(3), 340-355.

\bibitem{key3}
Lee, S. . Career Recommendation Systems using Collaborative Filtering. \emph{International Journal of Advanced Research in Computer Science}, 9(2), 456-465.

\end{thebibliography}

\end{document}